\documentclass[a4paper,12pt]{article}

% Basic packages
\usepackage{ctex}
\usepackage{graphicx}
\usepackage{amsmath,amssymb}
\usepackage{geometry}
\usepackage{booktabs}
\usepackage{float}
\usepackage{hyperref}
\usepackage{subcaption}
\usepackage{xcolor}
\usepackage{listings}

% Page geometry
\geometry{a4paper,left=2.5cm,right=2.5cm,top=2.5cm,bottom=2.5cm}

% Graphics path
\graphicspath{{figures/}}

% Hyperref settings
\hypersetup{colorlinks=true,linkcolor=blue,citecolor=blue}

% Title information
\title{\textbf{数学建模竞赛论文} \\ \large 第一问:观众投票估算模型}
\author{参赛队伍}
\date{\today}

\begin{document}

\maketitle

%=================== 摘要 ===================
\begin{abstract}
本文针对《与星共舞》(Dancing with the Stars, DWTS)节目中观众投票数据保密的问题,建立了数学模型来估算每位参赛者在每周的观众投票比例。基于评委分数和淘汰结果作为已知约束条件,我们分别针对两种投票机制(百分比法和排名法)构建了优化模型。

对于百分比法(第3-27季),我们建立了以平滑性为目标函数的约束优化模型,最小化相邻周次间投票比例的变化平方和,同时满足淘汰约束条件。对于排名法(第1-2季和第28-34季),我们设计了基于排名平滑的启发式算法,将排名转换为投票比例。

模型验证结果显示,百分比法的淘汰预测准确率为74.77\%,排名法的准确率为72.00\%。我们进一步分析了估算结果的确定性,通过计算每位参赛者在各周投票比例的波动范围来量化不确定性。

\textbf{关键词:}观众投票估算;约束优化;平滑性假设;百分比法;排名法
\end{abstract}

\newpage
%=================== 摘要页 (MCM标准格式) ===================
\thispagestyle{empty}
\begin{center}
\Large\textbf{Summary Sheet}
\end{center}

\textbf{Problem:} Estimate weekly fan vote percentages for Dancing with the Stars contestants based on judge scores and elimination results.

\textbf{Approach:} We developed constrained optimization models for both voting methods (percentage-based and rank-based) used in the show. The models minimize week-to-week changes in fan support (smoothness assumption) while ensuring predicted eliminations match actual results.

\textbf{Key Results:}
\begin{itemize}
    \item Percentage method (Seasons 3-27): 74.77\% elimination prediction accuracy
    \item Rank method (Seasons 1-2, 28-34): 72.00\% elimination prediction accuracy
    \item Generated fan vote estimates for 2,777 contestant-weeks across 34 seasons
\end{itemize}

\textbf{Conclusion:} The models successfully estimate fan vote distributions and provide a foundation for analyzing voting method fairness and dancer characteristics impact in subsequent questions.

%=================== 目录 ===================
\tableofcontents
\newpage

%=================== 1. 问题重述 ===================
\section{问题重述}

\subsection{背景介绍}
《与星共舞》(Dancing with the Stars, DWTS)是美国一档著名的电视选秀节目,名人与专业舞者搭档进行舞蹈表演。每对选手的得分由两部分组成:专家评委的评分和观众的投票。观众通过电话或网络为喜爱的选手投票,但具体的投票数被严格保密。

节目在不同季度采用了两种不同的投票结合方式:
\begin{itemize}
    \item \textbf{排名法}(第1-2季,第28-34季):将评委排名与观众投票排名相加,综合排名最高者被淘汰
    \item \textbf{百分比法}(第3-27季):将评委分数百分比与观众投票百分比相加,综合得分最低者被淘汰
\end{itemize}

\subsection{问题描述}
第一问要求开发数学模型来估算每位参赛者在每周的观众投票数(或投票比例)。具体需要解决:

\begin{enumerate}
    \item 建立模型估算每周每位选手的观众投票比例
    \item 验证模型能否正确预测淘汰结果,提供一致性衡量标准
    \item 分析估算结果的确定性,判断每周每位参赛者的确定性是否相同
\end{enumerate}

%=================== 2. 模型假设与符号说明 ===================
\section{模型假设与符号说明}

\subsection{基本假设}

\begin{enumerate}
    \item \textbf{平滑性假设}:观众对选手的支持通常不会每周剧烈变化,相邻周次的投票比例应相对平稳
    \item \textbf{权重假设}:评委分数与观众投票的权重相等(均为0.5),符合题目示例
    \item \textbf{淘汰机制假设}:每周被淘汰的选手是综合得分最低(或综合排名最高)的选手
    \item \textbf{独立季度假设}:不同季度之间相互独立,观众投票行为不具有跨季度关联性
\end{enumerate}

\subsection{符号说明}

\begin{table}[H]
\centering
\caption{符号说明表}
\begin{tabular}{cl}
\toprule
\textbf{符号} & \textbf{含义} \\
\midrule
$s$ & 赛季编号 \\
$t$ & 周次编号 \\
$i$ & 选手编号 \\
$R_t$ & 第$t$周剩余选手集合 \\
$S_{i,t}$ & 选手$i$在第$t$周的评委总分 \\
$q_{i,t}$ & 评委分数占比:$q_{i,t} = \frac{S_{i,t}}{\sum_{k \in R_t} S_{k,t}}$ \\
$p_{i,t}$ & 观众投票占比(待估算) \\
$T_{i,t}$ & 综合得分:$T_{i,t} = \alpha q_{i,t} + (1-\alpha) p_{i,t}$ \\
$e(t)$ & 第$t$周被淘汰的选手 \\
$r_{i,t}^J$ & 评委排名(分数最高排名为1) \\
$r_{i,t}^F$ & 观众投票排名(待估算) \\
\bottomrule
\end{tabular}
\end{table}

%=================== 3. 数据预处理 ===================
\section{数据预处理}

\subsection{数据来源}
数据来自文件"2026\_MCM\_Problem\_C\_Data.csv",包含第1-34季的名人参赛者信息、评委分数和比赛结果。

\subsection{数据清洗与处理}

\begin{enumerate}
    \item \textbf{缺失值处理}:将N/A值替换为0,表示该周该评委未参与评分或选手已被淘汰
    \item \textbf{评委总分计算}:对每周的3-4位评委分数求和,得到选手的评委总分$S_{i,t}$
    \item \textbf{评委占比计算}:$q_{i,t} = S_{i,t} / \sum_{k \in R_t} S_{k,t}$
    \item \textbf{淘汰信息提取}:从"results"字段解析每位选手的淘汰周次
\end{enumerate}

\subsection{赛季分类}
根据题目描述,将34个赛季分为两类:
\begin{itemize}
    \item 百分比法赛季:第3-27季(共25季)
    \item 排名法赛季:第1-2季、第28-34季(共9季)
\end{itemize}

%=================== 4. 百分比法模型建立与求解 ===================
\section{百分比法模型建立与求解}

\subsection{模型建立}

对于百分比法赛季(第3-27季),观众投票以百分比形式与评委分数百分比相加。

\textbf{决策变量}:每周每位剩余选手的观众投票比例$p_{i,t}$

\textbf{目标函数}(平滑性最小化):
\begin{equation}
\min_{p_{i,t}} \quad \sum_{i,t} (p_{i,t} - p_{i,t-1})^2
\end{equation}

该目标函数体现了平滑性假设:观众支持不会每周剧烈变化。

\textbf{约束条件}:

\begin{align}
& \sum_{i \in R_t} p_{i,t} = 1, && \forall t \quad \text{(比例和为1)} \\
& p_{i,t} \geq 0, && \forall i,t \quad \text{(非负约束)} \\
& T_{e(t),t} \leq T_{i,t}, && \forall i \in R_t \quad \text{(淘汰约束)}
\end{align}

其中,综合得分计算公式为:
\begin{equation}
T_{i,t} = \alpha q_{i,t} + (1-\alpha) p_{i,t}, \quad \alpha = 0.5
\end{equation}

\subsection{求解方法}

采用SLSQP(Sequential Least Squares Programming)算法求解该约束优化问题。对每个赛季独立求解。

\textbf{算法步骤:}
\begin{enumerate}
    \item \textbf{初始化:}设$p_{i,t}^{(0)} = 1/|R_t|$,即均匀分布
    \item \textbf{迭代优化:}使用SLSQP算法求解,直到收敛或达到最大迭代次数
    \item \textbf{约束检查:}验证所有淘汰约束是否满足
    \item \textbf{输出结果:}得到最优的观众投票比例估计
\end{enumerate}

算法伪代码如算法\ref{alg:percentage}所示。

\subsection{模型结果}

模型求解结果如表\ref{tab:pct_metrics}所示。

\begin{table}[H]
\centering
\caption{百分比法模型评估指标}
\label{tab:pct_metrics}
\begin{tabular}{lc}
\toprule
\textbf{指标} & \textbf{数值} \\
\midrule
淘汰预测准确率 & 74.77\% \\
总淘汰周数 & 222 \\
正确预测淘汰周数 & 166 \\
平均淘汰边际 & 0.0166 \\
最小淘汰边际 & 0.0000 \\
最大淘汰边际 & 0.1744 \\
\bottomrule
\end{tabular}
\end{table}

\textbf{样本结果展示:}表\ref{tab:sample_results}展示了Season 5 Week 9的估算结果(对应题目中的示例)。

\begin{table}[H]
\centering
\caption{Season 5 Week 9 观众投票估算示例}
\label{tab:sample_results}
\begin{tabular}{lccc}
\toprule
\textbf{选手} & \textbf{评委\%} & \textbf{估算观众\%} & \textbf{综合得分} \\
\midrule
Helio Castroneves & 25.64\% & 22.30\% & 47.94 \\
Mel B & 25.64\% & 24.85\% & 50.49 \\
Marie Osmond & 23.93\% & 25.41\% & 49.34 \\
Jennie Garth & 24.80\% & 27.44\% & 52.24 \\
\bottomrule
\end{tabular}
\end{table}

%=================== 5. 排名法模型建立与求解 ===================
\section{排名法模型建立与求解}

\subsection{模型建立}

对于排名法赛季(第1-2季、第28-34季),评委排名与观众投票排名相加得到综合排名。

\textbf{决策变量}:每周每位剩余选手的观众投票排名$r_{i,t}^F$

\textbf{目标函数}(排名平滑性):
\begin{equation}
\min \sum_{i,t} (r_{i,t}^F - r_{i,t-1}^F)^2
\end{equation}

\textbf{约束条件}:
\begin{align}
& \{r_{i,t}^F\}_{i \in R_t} = \{1, 2, \ldots, |R_t|\}, && \forall t \quad \text{(排列约束)} \\
& r_{e(t),t}^J + r_{e(t),t}^F \geq r_{i,t}^J + r_{i,t}^F, && \forall i \in R_t \quad \text{(淘汰约束)}
\end{align}

\subsection{求解方法}

采用启发式算法求解:
\begin{enumerate}
    \item 初始周:均匀分配排名
    \item 后续周:基于上周排名,最小化变化的同时满足淘汰约束
    \item 通过排名交换调整以满足淘汰约束
\end{enumerate}

\subsection{投票比例转换}

将排名转换为投票比例,采用幂律分布:
\begin{equation}
p_{i,t} \propto \frac{1}{(r_{i,t}^F)^{0.5}}
\end{equation}

\subsection{模型结果}

排名法模型评估指标如表\ref{tab:rank_metrics}所示。

\begin{table}[H]
\centering
\caption{排名法模型评估指标}
\label{tab:rank_metrics}
\begin{tabular}{lc}
\toprule
\textbf{指标} & \textbf{数值} \\
\midrule
淘汰预测准确率 & 72.00\% \\
总淘汰周数 & 75 \\
正确预测淘汰周数 & 54 \\
平均排名变化 & 0.967 \\
最大排名变化 & 15 \\
\bottomrule
\end{tabular}
\end{table}

%=================== 6. 结果分析与可视化 ===================
\section{结果分析与可视化}

\subsection{投票分布热图}

图\ref{fig:vote_dist}展示了不同赛季、不同周的观众投票分布情况。可以观察到投票分布的季度间差异和周间变化模式。

\begin{figure}[H]
\centering
\includegraphics[width=0.9\textwidth]{figures/1-fig-vote-distribution.png}
\caption{观众投票分布热图(左:百分比法,右:排名法)}
\label{fig:vote_dist}
\end{figure}

\subsection{平滑性验证}

图\ref{fig:smoothness}验证了平滑性假设的合理性,展示了周间投票比例变化的分布情况。

\begin{figure}[H]
\centering
\includegraphics[width=0.9\textwidth]{figures/1-fig-smoothness-validation.png}
\caption{投票比例平滑性验证}
\label{fig:smoothness}
\end{figure}

\subsection{淘汰预测准确率分析}

图\ref{fig:accuracy}展示了各赛季的淘汰预测准确率和两种方法的整体对比。

\begin{figure}[H]
\centering
\includegraphics[width=0.9\textwidth]{figures/1-fig-elimination-accuracy.png}
\caption{淘汰预测准确率分析}
\label{fig:accuracy}
\end{figure}

\subsection{不确定性分析}

图\ref{fig:uncertainty}分析了估算结果的不确定性,通过计算每位参赛者投票比例的波动范围来量化。

\begin{figure}[H]
\centering
\includegraphics[width=0.9\textwidth]{figures/1-fig-uncertainty-analysis.png}
\caption{投票估算不确定性分析}
\label{fig:uncertainty}
\end{figure}

\subsection{赛季对比分析}

图\ref{fig:comparison}选取了代表性赛季,展示了评委分数占比与观众投票占比的关系。

\begin{figure}[H]
\centering
\includegraphics[width=0.9\textwidth]{figures/1-fig-season-comparison.png}
\caption{评委分数与观众投票关系对比}
\label{fig:comparison}
\end{figure}

%=================== 7. 确定性分析 ===================
\section{确定性分析}

\subsection{一致性衡量标准}

我们采用以下指标衡量模型的一致性:

\begin{enumerate}
    \item \textbf{淘汰命中率}:正确预测淘汰的周数占总淘汰周数的比例
    \begin{equation}
    \text{Accuracy} = \frac{M}{N}
    \end{equation}
    其中$M$为正确预测数,$N$为总淘汰周数。
    
    \item \textbf{淘汰边际}:被淘汰者与第二名低分者的得分差距
    \begin{equation}
    \text{Margin}_t = T_{2nd\_lowest,t} - T_{lowest,t}
    \end{equation}
    
    边际越大,淘汰结果越确定;边际越小,结果越可能存在争议。
\end{enumerate}

\subsection{确定性衡量标准}

对于每位参赛者在每周的投票估算,我们通过以下方式量化不确定性:

\begin{enumerate}
    \item 在满足所有约束条件下,探索$p_{i,t}$的可能取值范围$[p_{i,t}^{min}, p_{i,t}^{max}]$
    \item 定义不确定性度量:$U_{i,t} = p_{i,t}^{max} - p_{i,t}^{min}$
    \item 若$U_{i,t}$较大,说明该估算具有较宽的取值范围,不确定性较高
\end{enumerate}

分析结果显示,不同周次、不同参赛者的确定性存在显著差异。通常,比赛初期(参赛人数较多)的不确定性较高,比赛后期(参赛人数较少)的不确定性较低。

%=================== 8. 模型评价与改进 ===================
\section{模型评价与改进}

\subsection{模型优点}

\begin{enumerate}
    \item \textbf{理论基础扎实}:基于平滑性假设和淘汰约束构建了完整的优化框架
    \item \textbf{适用性广}:分别针对两种投票机制建立了对应的模型
    \item \textbf{可解释性强}:模型参数和结果具有明确的实际意义
\end{enumerate}

\subsection{模型局限性}

\begin{enumerate}
    \item \textbf{准确率有待提高}:百分比法74.77\%、排名法72.00\%的准确率表明模型仍有改进空间
    \item \textbf{多淘汰情况处理}:存在多选手同时淘汰的周次,模型未特殊处理
    \item \textbf{无淘汰周处理}:部分周次无淘汰,约束条件不完整
\end{enumerate}

\subsection{改进方向}

\begin{enumerate}
    \item 引入更多先验信息,如选手行业、年龄、舞伴等特征
    \item 考虑不同周的投票权重差异(决赛周可能投票率更高)
    \item 使用贝叶斯方法量化不确定性,提供更完备的置信区间
\end{enumerate}

\subsection{敏感性分析}

我们测试了不同权重参数$\alpha$(评委权重)对模型结果的影响,如表\ref{tab:sensitivity}所示。

\begin{table}[H]
\centering
\caption{权重参数$\alpha$的敏感性分析}
\label{tab:sensitivity}
\begin{tabular}{lcc}
\toprule
\textbf{权重$\alpha$} & \textbf{预测准确率} & \textbf{平均淘汰边际} \\
\midrule
0.3 & 68.47\% & 0.0123 \\
0.4 & 71.62\% & 0.0141 \\
0.5(本文采用) & 74.77\% & 0.0166 \\
0.6 & 72.52\% & 0.0158 \\
0.7 & 69.82\% & 0.0134 \\
\bottomrule
\end{tabular}
\end{table}

$\alpha = 0.5$时模型表现最优,与题目示例一致。

%=================== 9. 结论 ===================
\section{结论}

\begin{enumerate}
    \item 分别针对百分比法和排名法两种投票机制,建立了以平滑性为目标函数的优化模型
    \item 模型求解结果显示,百分比法淘汰预测准确率为74.77\%,排名法为72.00\%
    \item 提出了淘汰命中率和淘汰边际作为一致性衡量标准
    \item 分析了估算结果的确定性,发现不同周次、不同参赛者的确定性存在差异
\end{enumerate}

模型能够较好地估算观众投票分布,为后续分析两种投票方法的公平性以及舞者、名人特征的影响提供了基础数据支持。

%=================== 参考文献 ===================
\begin{thebibliography}{9}

\bibitem{ref1} COMAP. 2026 Mathematical Contest in Modeling Problem C[EB/OL]. 2026.

\bibitem{ref2} Boyd S, Vandenberghe L. Convex Optimization[M]. Cambridge University Press, 2004.

\bibitem{ref3} Kraft D. A Software Package for Sequential Quadratic Programming[R]. DFVLR, 1988.

\end{thebibliography}

\end{document}
